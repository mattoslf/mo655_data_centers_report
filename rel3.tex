\documentclass[12pt,a4paper]{report}
\usepackage[hmargin=2cm,vmargin=3.5cm,bmargin=2cm]{geometry}
\usepackage[utf8]{inputenc}
\usepackage[portuguese]{babel}
\usepackage[T1]{fontenc}
\usepackage{amsmath}
\usepackage{amsfonts}
\usepackage{amssymb}
\usepackage{makeidx}
\usepackage{graphicx}
\usepackage{listings}
\usepackage{indentfirst}
\usepackage[pdftex]{hyperref}
\usepackage{csvsimple}
\usepackage{color}
\usepackage{mathtools}
\usepackage{float}

% Headers and Footers
\usepackage{fancyhdr}

\addto\captionsportuguese{\renewcommand{\chaptername}{}}

\definecolor{mygreen}{rgb}{0,0.6,0}
\definecolor{mygray}{rgb}{0.5,0.5,0.5}
\definecolor{mymauve}{rgb}{0.58,0,0.82}

\lstset{ %
  backgroundcolor=\color{white},   % choose the background color; you must add \usepackage{color} or \usepackage{xcolor}
  basicstyle=\footnotesize,        % the size of the fonts that are used for the code
  breakatwhitespace=false,         % sets if automatic breaks should only happen at whitespace
  breaklines=true,                 % sets automatic line breaking
  captionpos=b,                    % sets the caption-position to bottom
  commentstyle=\color{mygreen},    % comment style
  deletekeywords={...},            % if you want to delete keywords from the given language
  escapeinside={\%*}{*)},          % if you want to add LaTeX within your code
  extendedchars=true,              % lets you use non-ASCII characters; for 8-bits encodings only, does not work with UTF-8
  frame=single,                    % adds a frame around the code
  keywordstyle=\color{blue},       % keyword style
%  language=Octave,                 % the language of the code
  morekeywords={*,...},            % if you want to add more keywords to the set
  numbers=left,                    % where to put the line-numbers; possible values are (none, left, right)
  numbersep=5pt,                   % how far the line-numbers are from the code
  numberstyle=\tiny\color{mygray}, % the style that is used for the line-numbers
  rulecolor=\color{black},         % if not set, the frame-color may be changed on line-breaks within not-black text (e.g. comments (green here))
  showspaces=false,                % show spaces everywhere adding particular underscores; it overrides 'showstringspaces'
  showstringspaces=false,          % underline spaces within strings only
  showtabs=false,                  % show tabs within strings adding particular underscores
  stepnumber=2,                    % the step between two line-numbers. If it's 1, each line will be numbered
  stringstyle=\color{mymauve},     % string literal style
  tabsize=2,                       % sets default tabsize to 2 spaces
  title=\lstname                   % show the filename of files included with \lstinputlisting; also try caption instead of title
}


%\begin{itemize}
%\item A saída deve estar em ordem crescente (o elemento da saída não pode ser menor do que o seu antecessor de acordo com o critério adotado para ordenação);
%\item A saída é uma permutação (reordenação) da entrada;
%\end{itemize}
	
%\section{Algoritmos $\mathbf{O(n^2)}$}
%\subsection{Bubble Sort}

%\begin{lstlisting}[language=C++]
%\end{lstlisting}

%\begin{figure}[H]
%\includegraphics[width=\textwidth]{Imagens/bubble.png}
%\caption{Bubble Sort para diversos formatos de vetores}
%\end{figure}

%\begin{table}[H]
%\caption{Tabela dos tempos para os vetores de melhores e piores casos}
%\begin{center}
%\begin{tabular}{|c|c|c|c|c|c|}
%\hline 
%Caso & Bubble Sort (us) & Insertion Sort (us) & Merge Sort (us) & Heap Sort (us) & Quick Sort (us) \\ 
%\hline
%Melhor & 0,86	& 0,84 & 8,49 & 1,99 & 2,61 \\
%\hline 
%Pior & 24,97 & 24,56 & 8,97 & 7,73 & 15,8 \\
%\hline
%\end{tabular}
%\end{center}
%\end{table}


% Headers and Footers
\pagestyle{fancy} % fancy style
\lhead{\rightmark} % left header
\rhead{\leftmark} % right header
\lfoot{} % left footer
\cfoot{\textbf{\thepage}} % central footer
\rfoot{} % right footer
% Headers and Footers

\makeindex

% define the title
\author{Leandro Souza da  Silva - RA: 191082 \\ Luís Felipe Mattos - RA: 107822}
\title{MO655 - Comunicação em Datacenters}
%\date{}
\begin{document}

% generates the title
\maketitle

% insert the table of contents
\tableofcontents

\chapter{Introdução}

Com o crescimento da demanda dos usuários por poder computacional e armazenamento, cada vez mais as grandes empresas estão investindo em estruturas próprias de datacenters. Esta estrutura inclui tanto os computadores em si, os discos de armazenamento e os racks como também inclui a própria sala que ficarão estes racks. Estas salas devem tem uma arquitetura própria, como por exemplo, o piso elevado, sistema de refrigeração e circulação de ar. Um exemplo por ser visto na figura \ref{dc_overview}.\\

\begin{figure}[H]
\centering
\includegraphics[width=.7\textwidth]{imagens/datacenter.png}
\caption{Visão geral de um Datacenter}
\label{dc_overview}
\end{figure}

Além da estrutura física, a parte computacional, principalmente relacionada à comunicação interna e externa do datacenter possui alguns requisitos básicos para que possa oferecer um serviço de qualidade para os usuários. Estes requisitos são citados a seguir:\\

\begin{itemize}
\item Escalabilidade
\item Tolerância a Falhas
\item Latêcia
\item Capacidade da Rede
\item Virtualização
\end{itemize}

A seguir, os requisitos citados serão mais detalhados.\\

\section{Requisitos de Rede}

\subsection{Escalabilidade}
O sistema deve ser construido de tal forma que seja possível haver uma expansão, caso a demanda aumente. Este requisito diz respeito tanto ao hardware como ao software. Para o hardware, a estrutura das máquinas deve permitir que o sistema seja melhorado e também deve haver espaço físico para a inclusão de novas máquinas. Atualmente, existem alguns sistemas modulares que possuem uma fácil integração de novos módulos.\\

Um exemplo é a utilização de datacenters em containers, cada container possui um sistema completo com refrigeração própria e é facilmente transportado. Com isso, pode-se expandir facilmente uma estrutura de um datacenter. Um exemplo de container pode ser visto na figura \ref{dc_container}.\\

\begin{figure}[H]
\centering
\includegraphics[width=.7\textwidth]{imagens/container_dc.jpg}
\caption{Datacenter em container}
\label{dc_container}
\end{figure}

\subsection{Tolerância a Falhas}
O sistema deve ser capaz de prevenir e corrigir falhas. Por causa disso, a maioria dos sistemas de datacenters possuem redundâncias em quase todos aspectos do datacenter. Existem backups dos dados dos usuários, a comunicação interna é feita de modo que existam vários caminhos possíveis da fonte para o destino e além disso, alguns datacenters possuem backups em outros datacenters. Apesar disso, existe o custo de manter estas cópias atualizadas.\\

Mais a frente falaremos um pouco mais sobre as redundâncias dos caminhos de comunicação interna dos datacenters, tanto relacionados à topologia como relacionado aos protocolos de comunicação e roteamento.\\

\subsection{Latêcia}
Um dos principais desafios dos datacenters é possuir baixa latência, assim, a performance do sistema como um todo se mantém em um nível aceitável pelos usuários. Para isso, a topologia é muito importante, uma vez que quanto menor o caminho entre a fonte e o destino, menor a latência. Porém, outro fator que influencia muito a latência é o nível de congestionamento da rede, mais a frente iremos tratar sobre os protocolos e como estes controlam o nível de congestionamento da rede.\\

\subsection{Capacidade da Rede}


\subsection{Virtualização}


\chapter{Motivação}

\chapter{Topologias}

\chapter{Protocolos}

\chapter{Tendências}

\chapter{Conclusão}

\chapter{Referências}

\begin{itemize}
\item Manual de referência do NS-3 \begin{verbatim}https://www.nsnam.org/docs/release/3.8/manual.pdf\end{verbatim}
\item Documentação \begin{verbatim}https://www.nsnam.org/doxygen/index.html\end{verbatim}
\end{itemize}

\end{document}
